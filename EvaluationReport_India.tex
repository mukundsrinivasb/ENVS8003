\documentclass[12pt]{article}

% Packages
\usepackage[utf8]{inputenc}
\usepackage{graphicx}
\usepackage{hyperref}
\usepackage{enumitem}
\usepackage[backend=biber,style=authoryear,citestyle=authoryear,natbib=true]{biblatex}
\addbibresource{references_india.bib}
\usepackage[a4paper, margin=0.8in]{geometry}
\usepackage{titlesec}
\setlength{\parskip}{1em} 
\setlength{\bibitemsep}{1em} % Add spacing between bibliography entries

%Customizing section titles
% Customize section and subsection titles
\titleformat{\section}
  {\normalfont\Large\bfseries} % Section titles will be in large, bold font
  {\thesection}{1em}{}
%   \titlespacing{\subsection}{0pt}{*0}{-0.8em}
  [\titlerule]
  
%Italicised subsectiom
\titleformat{\subsection}
  {\normalfont\bfseries\itshape} % Subsection titles will be slightly smaller
  {\thesubsection}{1em}{}
\titlespacing{\subsection}{0pt}{0em}{-0.3em}

% Regular non-italic subsection (unnumbered)
\newcommand{\regularsubsection}[1]{%
  \vspace{1em} % Adjust spacing if needed
  \noindent\textbf{#1}\par\vspace{-0.3em}}

% Title
\title{Adaptation Plan Evaluation Report}
\author{Mukund Balaji Srinivas | u7274095}
\date{} % Remove the date 

\begin{document}

% Title Page
\maketitle

\begin{abstract}
This section provides a brief summary of the evaluation report, outlining the key question being addressed, methods used, 
and main conclusions.
\end{abstract}

\newpage

\section*{Introduction}
Climate change in India affects multiple facets of the ecosystem, including oceans (\cite{marathe_2021}).
coastal regions (\cite{gupta_impact_2019}), water resources (\cite{shiva_shankar_2021}) and forests (\cite{Lele2019ClimateCA}). Additionally, it significantly impacts 
agriculture (\cite{Kumar2023DeterminantsOC}), urban areas, public health(\cite{rajput_2022}), and energy infrastructure (\cite{Yarlagadda_22}), 
creating complex challenges that require coordinated responses. Each of these vulnerabilities will be examined in detail to highlight the specific risks.
\subsection*{Coastal Regions}
Climate change is projected to exacerbate the impacts of tropical cyclonic storms by increasing their intensity as sea surface temperatures 
rise. The North Indian Ocean comprising the Bay of Bengal and the Arabian Sea accounts for only 7\% of global cyclones 
these storms are disproportionately destructive, particularly along the densely populated, low-lying East Indian and Bangladeshi coasts, 
which are highly vulnerable to storm surges and flooding. Also, while the Arabian Sea has historically experienced fewer high-intensity storms due to 
factors such as unfavourable wind shear, dry air from the Thar Desert, and cooler sea temperatures, recent years have seen a rise in stronger cyclones 
in the region. This shift suggests that changing climatic conditions are altering traditional storm patterns and intensifying their impacts, posing 
new challenges for coastal resilience and disaster preparedness (\cite{gupta_impact_2019})

\subsection*{Water resources}
\regularsubsection{Adaptation Policy and activities}
The National Action Plan on Climate Change [NAPCC] (\cite{napcc_2008}) articulates India's adaptation goals addressing climate 
vulnerabilities across key sectors while promoting sustainable growth. NAPCC identifies eight National Missions, which form 
the core of the National Action Plan, out of which five missions are focusing on adaptation, which are:
\begin{enumerate}[topsep=0pt, parsep=0pt, partopsep=0pt]
  \item \textbf{National Water Mission}: Promotes integrated water resource management to conserve water, minimize wastage, 
  and ensure equitable distribution across regions and states.  
  \item \textbf{Mission for Sustaining the Himalayan Ecosystem}: Aims to safeguard Himalayan glaciers and ecosystems through 
  sustainable management and a monitoring network.  
  \item \textbf{National Mission for a Green India}: Focuses on enhancing ecosystem services, including carbon sequestration, 
  through afforestation and reforestation efforts.
  \item \textbf{National Mission for Sustainable Agriculture}: Seeks to build climate resilience in agriculture by adopting new 
  technologies, integrating traditional knowledge, and introducing improved credit and insurance systems. 
  \item \textbf{National Mission on Strategic Knowledge for Climate Change}: Facilitates research, innovation, 
  and global collaboration to address climate challenges and develop effective responses.
\end{enumerate}

\end{document}
