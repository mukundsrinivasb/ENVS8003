\documentclass[12pt]{article}

% Packages
\usepackage[utf8]{inputenc}
\usepackage{graphicx}
\usepackage{hyperref}
\usepackage[backend=biber,style=authoryear,citestyle=authoryear,natbib=true]{biblatex}
\addbibresource{references.bib}
\usepackage[a4paper, margin=1in]{geometry}
\setlength{\parskip}{1em} 
\setlength{\bibitemsep}{1em} % Add spacing between bibliography entries

% Title
\title{Adaptation Plan Evaluation Report}
\author{Mukund Balaji Srinivas | u7274095}
\date{} % Remove the date 

\begin{document}

% Title Page
\maketitle

\begin{abstract}
This section provides a brief summary of the evaluation report, outlining the key question being addressed, methods used, and main conclusions.
\end{abstract}

\newpage

\section*{Introduction}
% \subsection*{Background}

The World Economic Forum (\cite{masterson_2024})  outlines six key technologies critical 
for climate change adaptation, including the powerful roles of Earth Observation [EO] 
and Artificial Intelligence [AI]. EO technologies allow for real-time monitoring of 
environmental changes, such as deforestation, urbanization, and rising sea levels, offering 
critical data to enhance climate resilience (\cite{anderson_2017}). AI can help process this vast amount of EO data, identify patterns, and generate predictive models, which can support better decision-making in areas like disaster risk management and agricultural planning (\cite{Huntingford_2019}).

However, the significant role that Information and Communication Technology [ICT] 
plays in global emissions—over 3 percent of carbon emissions (\cite{jones_2018}) . Specifically , 
companies like Samsung and Amazon had a footprint of around 40,000MT of \(CO_2\) equivalent in 2021 (\cite{navarro_2023_the}) —raises the question of how big tech companies are addressing their environmental impact. While AI and EO tools offer potential benefits in mitigating climate change, their carbon costs are often overlooked. For instance, AI models require massive computational power, and EO platforms consume vast amounts of energy through cloud-based infrastructure (\cite{taddeo_2021}).

Given the dual role of ICT as both a significant contributor to global carbon emissions and a 
key driver of climate change solutions, it is crucial to examine the adaptation strategies 
adopted by tech companies to mitigate their environmental impact. This report systematically 
evaluates the sustainability reports of major tech companies—Meta, Apple, Amazon, Google, Nvidkia, 
Samsung and Dell—focusing on their key businesses, associated climate risks, and summaries of 
their adaptation strategies. While technologies like  AI and EO are instrumental in addressing 
climate change, their environmental costs must be carefully weighed against their benefits. Currently, no specific 
framework exists to evaluate these strategies in the tech sector, necessitating reliance on established climate change 
metrics. To this effect, the techniques outlined in the framework provided by the International Institute for Environment and Development will guide the evaluation, ensuring that these companies’ adaptation policies align with broader sustainability goals. The evaluation will utilize the scorecard suggested by the IIED (\cite{craft_2016}) for systematic reporting, along with the framework proposed in the article by Jones N. (\cite{jones_2001}) 
to assess the adaptation strategies of the selected companies effectively.


\section*{Methods}
Describe the criteria being used for the evaluation in this section. Include a brief overview of the selected nation/region/sector and explain why this particular area was chosen for the evaluation.

\section*{Comparison}
Here, make the necessary comparisons based on the criteria from the Methods section. Discuss how different nations/regions/sectors perform according to these criteria. You may break this into subsections if there are multiple comparisons.

\section*{Discussion}
Synthesize the information you gathered from the comparisons. Discuss patterns, insights, and important findings based on the comparisons made.

\section*{Conclusion}
Summarize the main conclusions of your evaluation. Reflect on what the comparisons suggest about the criteria and any recommendations based on the results.

\newpage
\printbibliography

\end{document}

%Changes more 'K