\documentclass[12pt]{article}

% Packages
\usepackage[utf8]{inputenc}
\usepackage{graphicx}
\usepackage{hyperref}
% \usepackage{cite} % You can modify this for Fenner School Harvard referencing style

% Title
\title{Your Evaluation Report Title}
\author{Mukund Balaji Srinivas | u7274095}

\begin{document}

% Title Page
\maketitle

\begin{abstract}
This section provides a brief summary of the evaluation report, outlining the key question being addressed, methods used, and main conclusions.
\end{abstract}

\newpage

\section*{Introduction}
State the question being addressed in the report and provide context to the problem using relevant background information. This section should introduce the topic and explain why it is significant.

\section*{Methods}
Describe the criteria being used for the evaluation in this section. Include a brief overview of the selected nation/region/sector and explain why this particular area was chosen for the evaluation.

\section*{Comparison}
Here, make the necessary comparisons based on the criteria from the Methods section. Discuss how different nations/regions/sectors perform according to these criteria. You may break this into subsections if there are multiple comparisons.

\section*{Discussion}
Synthesize the information you gathered from the comparisons. Discuss patterns, insights, and important findings based on the comparisons made.

\section*{Conclusion}
Summarize the main conclusions of your evaluation. Reflect on what the comparisons suggest about the criteria and any recommendations based on the results.

\newpage
% \bibliographystyle{plain} % Change this to Fenner School Harvard style as required.
% \bibliography{references}

\end{document}

%Changes more '